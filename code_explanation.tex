\documentclass[letterpaper, a4paper]{article}

\usepackage{geometry}

\usepackage{minted}
\usepackage{mathtools}


\title{Simplex - Code Explanation}
\author{Ryan Horrell}
\date{November 2024}

\begin{document}

\section{Code Explanation}

\subsection{Functions}
This section will explain the use of functions in the order in which they are ran in the program.
\subsubsection{start()}
$start$, is the entry point for the program and, for the purposes of the coursework, stands only to define the variables necessary to compute the given problem.
\newline

Maximise $z = 2x_1 + 3x_2 + 4x_3 +x_4 + 8x_5 + x_6$ subject to
\begin{align*}
x_1-x_2+2x_3+x_5+x_6 & = 18\\
    x_2-x_3+x_4+3x_6 & \le 8\\
x_1+x_2-3x_3+x_4+x_5 & \le 36\\
     x_1-x_2+x_5+x_6 & \le 23\\
\end{align*}

with all variables non-negative: $x_1,\ x_2,\ x_3,\ x_4,\ x_5,\ x_6\ge 0$
\newline
\newline
Which is represented as,
\begingroup
\fontsize{8}{10}\selectfont
\begin{minted}{python}
which = 'max'
A = np.array([[1., -1.,  2.,  0.,  1.,  1.,  0.,  0.,  0.,  1.],
              [0.,  1., -1.,  1.,  0.,  3.,  1.,  0.,  0.,  0.],
              [1.,  1., -3.,  1.,  1.,  0.,  0.,  1.,  0.,  0.],
              [1., -1.,  0.,  0.,  1.,  1.,  0.,  0.,  1.,  0.]])
b = np.array([18., 8., 36., 23.])
c = np.array([2.+0.j, 3.+0.j, 4.+0.j, 1.+0.j, 8.+0.j, 1.+0.j, 0.+0.j, 0.+0.j, 0.+0.j, 0.-1.j])

\end{minted}
\endgroup
This formatting has been taken as the output of the function $init\_params$ and immediately given to the function $simplex$.
\newline

For the general use purpose of the code, $start$ functions differently, and requires sections to be commented and uncommented. If in this state the program first calls $get\_input$ 

\subsubsection{get\_input()}
$get\_input()$ which as you would expect, gets inputs from the user for a given problem. First asking to maximise or minimise. The program then asks for the number of variables and constraints, then asking for the coefficients to the objective function, then asking for the coefficients, constraint type, $\le,\ =,\ \ge$ and RHS value, for each constraint. Then returns the coefficients of the objective function, a 2D array of constraint variables, and dictionary of constraint type and RHS value.
\newline

These variables are given to $init\_params$.

\subsubsection{init\_params(A, b, c, which)}

$init\_params$ initialises the tableau along with other variables to a state the simplex method can start from. This begins by handling negative RHS constraint values, and updating constraint types, after which figures out which constraints need slack and/or artificial variables and concatenates this with the constraints matrix to create the initial tableau. Slack and artificial variables are represented with complex numbers, slack use real, artificial use pure imaginary as to differentiate them within the code and allow checks against Big-M values. Finally, if minimising, swap the parity of the objective function.
Returning the tableau, RHS bounds as values, and objective function coefficients.
\newline

At this point the variables are in a state where $simplex$ can work from.

\subsubsection{simplex(A, b, c, verbose=False)}
$simplex$ begins by running checks to see if the variables are in fact in an operable state. An error is thrown if the dimensions are incompatible.

\begingroup
\fontsize{8}{10}\selectfont
\begin{minted}{python}
# Check types are correct
if not isinstance(A, np.ndarray):
    A = np.array(A, dtype=np.float64)
if not isinstance(b, np.ndarray):
    b = np.array(b, dtype=np.float64)
if not isinstance(c, np.ndarray):
    c = np.array(c, dtype=np.complex128)

# Ensure Dimensions are Compatible
assert A.shape[0] == b.shape[0]
assert A.shape[1] == c.shape[0]

\end{minted}
\endgroup


The index of artificial variables, and basis are found as the standard columns of the tableau. Removing basic variable columns if they are also standard in the given problem. These are then ordered by where the 1 appears in the column and pValues are set to the RHS values to begin with. The BFS can be extracted from the information in the code, it is assumed that our BFS would be computed from setting the basic variables to zero, and finding a solution that way. Indeed here we extract the non-basic variables, and the order in which they appear in the variable $basisIndex$ defines the BFS. For our given problem, it would become:

\begin{center}
$a_1 = 18,\ s_1 = 8,\ s_2 = 36,\ s_3 = 23$
\end{center}
Which is represented in the initial tableau:
\[
\begin{array}{|c|c|cccccccccc|c|}
\hline
& c_j & 2 & 3 & 4 & 1 & 8 & 1 & 0 & 0 & 0 & -M \\
\hline
c_b & \text{Base} & x_1 & x_2 & x_3 & x_4 & x_5 & x_6 & s_1 & s_2 & s_3 & a_1 & \text{R} \\
\hline
-M & a_1 & 1 & -1 & 2 & 0 & 1 & 1 & 0 & 0 & 0 & 1 & 18 \\
0 &  s_1 & 0 & 1 & -1 & 1 & 0 & 3 & 1 & 0 & 0 & 0 & 8 \\
0 &  s_2 & 1 & 1 & -3 & 1 & 1 & 0 & 0 & 1 & 0 & 0 & 36 \\
0 &  s_3 & 1 & -1 & 0 & 0 & 1 & 1 & 0 & 0 & 1 & 0 & 23 \\
\hline
& Z & -M-2 & M-3 & -2M-4 & -1 & -M-8 & -M-1 & 0 & 0 & 0 & 0 & -18M \\
\hline
\end{array}
\]

\begingroup
\fontsize{8}{10}\selectfont
\begin{minted}{python}
# Artificial Variables use imaginary values to represent Big M
artificialBasisIndex = np.where(c.imag != 0)

# Basis Index finds columns that are standard, i.e. artificial/slack with single 1 and zeros elsewhere
basisIndex = np.where(
    (np.sum(A == 1, axis=0) == 1) & (np.sum(A == 0, axis=0) == A.shape[0] - 1)
)[0]
basicIndex = np.array(range(c[c.real != 0].shape[0]))

# Remove basic variables from basis if present, case where basic variables are standard
basisIndex = basisIndex[~np.isin(basisIndex, basicIndex)]

# Get the index of the first 1 in each row
first_indices = [
    np.where(row == 1)[0][0] if np.any(row == 1) else len(row)
    for row in A.T[basisIndex]
]

# Get the order of rows based on the first occurrence of 1
order = np.argsort(first_indices)
basisIndex = basisIndex[order]
\end{minted}
\endgroup



Now the iterative process begins. The code finds the coefficients of the basis values, then the max P value by taking the dot product of pValues with the basis coefficients, next the evaluation variables are computed as the dot product of basis coefficients and each column, subtracting the associated coefficient of the objective function for that column.
To account for Big-M values in our choice of pivot column, complex numbers are used to represent M in the code,


\begingroup
\fontsize{8}{10}\selectfont
\begin{minted}{python}
# Magnitude of evaluation variables,
evalVarsMagn = np.sum((evalVars.real, evalVars.imag * 1e16), axis=0)

# Entering Variable
pivColIndex = np.argmin(evalVarsMagn)
\end{minted}
\endgroup
multiplying the complex part of each evaluation variable with a large number, then taking the minimum to give us our pivot column.

Now before finding our pivot row, we perform our checks for infeasibility and unboundedness.

\begingroup
\fontsize{8}{10}\selectfont
\begin{minted}{python}
# Entering Variable
pivColIndex = np.argmin(evalVarsMagn)
# Unboundedness check, if all values in the pivot column are less than or equal to zero the problem is unbounded
if (A[:, pivColIndex] <= 0).all():
    finalCoeff = np.zeros(c.shape[0])
    finalCoeff[basisIndex] = pValues
    m = c[c.real > 0].shape[0]
    coeffStr = [
        f"{f'X{i+1}' if i < m else f'A{c.shape[0]-i}' if c[i].imag != 0 else f'S{i-m + 1}'}"
        for i in range(0, c.shape[0])
    ]
    raise OverflowError(coeffStr[pivColIndex])
# Infeasibility check
# If all Evaluation Variables are greater then or equal to 0
if (evalVarsMagn >= 0).all():
    # If Artificial Variable are contained in the final solution the problem is infeasible
    artBasis = basisIndex[
    np.where(np.isin(basisIndex, artificialBasisIndex))
    ]
    infeasibilityCheck = (A[:, artBasis] > 0).any()
    if infeasibilityCheck:
        raise ValueError(np.where(artBasis == artificialBasisIndex)[0])
    pValues = {p + 1: pValues[i] for i, p in enumerate(basisIndex)}
    break
\end{minted}
\endgroup

For Unboundedness, if all values in the pivot column are non-positive then the problem is unbounded, and the function throws a OverflowError with the variable name as a parameter, which is printed nicely to the user.
For Infeasibility, if all evaluation variables are positive, we need to check whether artificial variables are contained in the final solution. If there is, an ValueError is thrown along with the variable causing the problem.

If the evaluation variables still contain non-positive values, continue by finding the pivot row. Which is done by finding to minimum positive value of pValues divided by each column. In the code negatives are ignored by setting them to infinity.
Now we have both pivot column, row and element, we can perform the row operations necessary for the next iteration of the tableau.

\begingroup
\fontsize{8}{10}\selectfont
\begin{minted}{python}
# Calculate Next Tableau Iteration
pValues[pivRowIndex] = pValues[pivRowIndex] / pivElement
pValueIndex[pivColIndex + 1] = pValues[pivRowIndex]
if pivColIndex in basisIndex:
    c += -c[pivColIndex] * pivRow
for i in range(A.shape[0]):
    if i != pivRowIndex:
        pValues[i] = (oldP[i] * pivElement - pivCol[i] * p1) / pivElement
        pValueIndex[i + 1] = pValues[i]
        for j in range(A.shape[1]):
            A[i, j] = (A[i, j] * pivElement - pivCol[i] * pivRow[j]) / pivElement
# Entering replaces Leaving variable
basisIndex[pivRowIndex] = pivColIndex
\end{minted}
\endgroup
This code performs the row operations for the next tableau iteration, and updates the basis indexes by swapping the pivot row (leaving) with pivot column (entering). The next iteration begins.

When the evaluation variables are all positive, and after checking for artificial variables in the final solution. We have our optimum value. This along with the final pValues are returned to be printed.

\end{document}
